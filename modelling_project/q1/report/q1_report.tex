% Created 2020-06-14 Sun 22:00
% Intended LaTeX compiler: pdflatex
\documentclass[11pt]{article}
\usepackage[utf8]{inputenc}
\usepackage[T1]{fontenc}
\usepackage{graphicx}
\usepackage{grffile}
\usepackage{longtable}
\usepackage{wrapfig}
\usepackage{rotating}
\usepackage[normalem]{ulem}
\usepackage{amsmath}
\usepackage{textcomp}
\usepackage{amssymb}
\usepackage{capt-of}
\usepackage{hyperref}
\date{\today}
\title{}
\hypersetup{
 pdfauthor={},
 pdftitle={},
 pdfkeywords={},
 pdfsubject={},
 pdfcreator={Emacs 26.3 (Org mode 9.1.9)}, 
 pdflang={English}}
\begin{document}

\tableofcontents

Tarik Pecaninovic 26946831

Question 1

\section{}
\label{sec:orgdb32c56}
\subsection{a)}
\label{sec:orge40f223}

We are given three functions which describe the derivative of three functions, x,y, and z, with respect to time (i.e., differential equations). 
The three functions are coupled in that they depend on each other and not time. To solve the system (given \(r\) and \(s\)) using RK2, at each time step
we approximate the values of x,y, and z using the previously computed values and the values of the derivatives evaluated at these previously computed
values (i.e., we use (x\(_{\text{i}}\),y\(_{\text{i}}\),z\(_{\text{i}}\)) and x', y', z' evaluated at this point to compute (x\(_{\text{i+1}}\),y\(_{\text{i+1}}\),z\(_{\text{i+1}}\)). The code implementing this (and which produced
the figures below) is in the script q1\_ rockpaperscissors\_ RK2.py


We set \(r=-1\) and \(s=1\) and plot for the intial conditions v\(_{\text{1}}\) = (0.2,0.3,0.5), v\(_{\text{2}}\) = (0.3,0.2,0.5), v\(_{\text{3}}\) = (0.2,0.5,0.3), v\(_{\text{4}}\) = (0.5,0.3,0.2), v\(_{\text{5}}\) = (0.3,0.5,0.2), and v\(_{\text{6}}\) = (0.5,0.2,0.3)
(figures shown in the files vi\_ A\_ plot.png where 1\(\le\) i \(\le\) 6). Note that the initial conditionas above account for all permutations of the sizes of the numbers in each coordinate. For
example v\(_{\text{1}}\) has x\(_{\text{0}}\) the smallest, y\(_{\text{0}}\) in-between and z\(_{\text{0}}\) the largest, v\(_{\text{4}}\) is reversed, etc.

From the figures we observe that the six intial conditions can fall into one of two types; where two initial conditions being in the same type implies they have the same overall behaviour.
We see in v\(_{\text{2}}\), v\(_{\text{3}}\), and v\(_{\text{4}}\) the variable corresponding to the smallest and largest value decreases linearly, while the other variable increases linaerly, until a stationary state is reached
and all three become constant. In the other case though we have the reverse occuring (e.g., the variable with the largest intial condition decreases linearly until its stationary value is reached).



\subsection{b)}
\label{sec:org0828440}

Repeating the above but with r=1, s=1 first and then r=-1,s=-1 we obtain the plots vi\_ B\_ pos\_ plot.py and vi\_ B\_ neg\_ plot.py  for 1\(\le\) i\(\le\) 6.
We observe that when r+s>0, all three variables decrease linearly until a stationary state is reached. Moreover, the values of each of the variables
in the stationary state depend on not just the the initial values, but the ordering of the sizes of the intial values. For example, we see that v2 to v4
are the same up to permutation of variables and similarly for v1,v5, and v6. \\
When r+s<0 we observe a similar situation as to the above except that the variables are linearly increasing instead of decreasing.
\end{document}